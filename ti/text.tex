\documentclass{article}
\usepackage[utf8]{inputenc}
\usepackage{listings}

\title{TI Referát}
\author{Filip Peterek}
\date{Listopad 2021}

\begin{document}

\maketitle

\section{CNF}

Startovací neterminál lze rozepsat na $\varepsilon$, poté
se však startovací neterminál nesmí objevit na pravé straně
žádného pravidla.

Jinak lze všechny neterminály rozepsat pouze na jeden
terminál nebo dva jiné neterminály.

$S \rightarrow \varepsilon$

$A \rightarrow BC \: | \: CD \: | \: a \: | \: b$

\section{Algoritmus Cocke-Younger-Kasami}

Dynamické programování -- přístup zdola nahoru (tabulation).
Algoritmus CYK využívá vlastností CNF. 

Přístup shora dolů by byl neefektivní - při rozpisu počátečního
neterminálu bychom museli vyzkoušet vysoké množství kombinací.
Přístup zdola nahoru nám umožní omezit množství kombinací, které
musíme vyzkoušet, tím, že nejprve přepíšeme terminály na neterminály,
a poté pouze hledáme pravidla, která dané neterminály mohly vytvořit.

Nejprve se všechny terminály přepíšou na odpovídající
neterminály (jeden terminál může odpovídat více neterminálům).
Tyto neterminály se zapíší do~tabulky jako substringy délky 1.

Poté se prochází n-písmenné substringy ($n \in 2..length$)
původního slova a hledají se pravidla, která odpovídají daným
dvěma neterminálům (terminály byly eliminovány v prvním kroku a
nahrazeny neterminály). Začíná se u substringů délky 2, délka
substringů poté lineárně roste. Využívá se toho, že v~CNF
lze neterminál přepsat pouze na dva jiné neterminály,
proto se substringy délky n rozdělí vždy jedině na dvě části
((1, n-1), (2, n-2), (3, n-3), ..., (n-1, 1)). Takto se nám
substring rozdělí na dva kratší substringy -- ovšem tyto
kratší substringy již máme zpracované a zaznamenané v tabulce
-- proto dynamické programování (tabulation). Proto se podíváme
pouze do tabulky, zda lze danou část substringu vygenerovat z některého
z neterminálů gramatiky. Pokud ne, nelze ani původní substring
vygenerovat pomocí současného dělení substringu a musíme zkoušet
jiné rozdělení momentálně zkoušeného substringu. Takto postupujeme
až dokud nevyzkoušíme substring délky n (tedy celé vstupní slovo).
Pokud se v množině neterminálů generujících původní slovo nachází
startovací neterminál, můžeme říct, že $w \in 
\mathcal{L}(G)$.

\section{Počet derivačních stromů}

Pro zjištění počtu derivačních stromů stačí již pouze jednoduchá
úprava algoritmu CYK.

\subsection{Generování derivačních stromů}

První možností, jak zjistit počet derivačních stromů, je vygenerovat
všechny derivační stromy a poté je pouze spočítat. Tato možnost
je flexibilnější, neboť umožňuje také získání samotných derivačních
stromů, má ale vyšší paměťové i výkonové nároky, protože vyžaduje
ukládání stromů v paměti a jejich následnou traverzaci.

Jádro algoritmu spočívá v ukládání derivačního podstromu pro každý
neterminál. Pro každý neterminál si vytvoříme derivační strom,
ve kterém si uložíme, ze kterých podstromů daný uzel vznikl.
Do tabulky si poté místo neterminálů ukládáme přímo tyto podstromy.

Díky tomu, že je gramatika v CNF, bude strom vždy binární.

\begin{lstlisting}[language=Scala]
case class ParseNode(
  nt: NonTerminal, 
  subnodes: (ParseNode, ParseNode)
)
\end{lstlisting}

Tímto způsobem můžeme vygenerovat více stromů pro jeden neterminál,
pokud by daný neterminál šlo vytvořit pomocí více pravidel. Alternativně
můžeme do derivačního stromu uložit seznam všech možných párů podstromů.

\begin{lstlisting}[language=Scala]
case class ParseNode(
  nt: NonTerminal, 
  subnodes: Seq[(ParseNode, ParseNode)]
)
\end{lstlisting}

Toto řešení je ekvivalentní, musíme však počítat s tím, že sekvence
subnodes obsahuje sadu nezávislých dvojic pravidel.

Aplikací algoritmu CYK takto získáme podobnou tabulku. Zda
$w \in \mathcal{L}(G)$ zjistíme podle toho, že se na vrcholu tabulky
nachází strom, jehož kořenem je počáteční neterminál.

Počty derivačních stromů poté získáme jednoduše -- pro první případ
stačí spočítat počet stromů s počátečním neterminálem na vrcholu tabulky.

Pro druhý případ stačí rekurzivně projít strom, jehož kořenem je počáteční
neterminál, pro každou dvojici vynásobit počet derivačních podstromů pro
levý a pravý neterminál (protože podstromy ve dvojici na~sobě závisí),
a sečíst počty derivačních stromů pro všechny dvojice (protože
jednotlivé dvojice na~sobě nezávisí).

\subsection{Počítání derivačních stromů bez jejich generování}

Pokud bychom nechtěli derivační stromy generovat, stačí nám místo
derivačního stromu pro konkrétní neterminál uložit dvojici obsahující
daný neterminál a počet pravidel, kolik daný neterminál na konkrétním místě
dokáže vygenerovat (protože negenerujeme slova, ale jdeme opačným směrem).
Aplikujeme podobný algoritmus výpočtu množství derivačních stromů, jako
při předchozí možnosti, nyní ovšem množství derivačních stromů počítáme
již při aplikaci algoritmu CYK. Počty derivačních stromů v rámci jedné
dvojice násobíme, počty pro~jednotlivé dvojice v rámci sekvence dvojic
sčítáme. Pokud poté na vrcholu tabulky získáme počáteční neterminál,
celkový počet derivačních stromů slova $w$ v gramatice $G$ je roven
číslu ve dvojici s tímto neterminálem.

\end{document}

